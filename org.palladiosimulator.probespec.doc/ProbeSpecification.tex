%% LaTeX-Beamer template for KIT design
%% by Erik Burger, Christian Hammer
%%
%% version 1.2
%%
%% mostly compatible to KIT corporate design v1.2
%% http://www.uni-karlsruhe.de/download/uka/Gestaltungsrichtlinien_komplett.pdf
%%
%% Problems, bugs and comments to
%% burger@ipd.uka.de

\documentclass[18pt]{beamer}
\usetheme{kit}

% remove the following line for German date style and logos

% if a custom picture is to be used on the title page, copy it into the 'logos'
% directory, in the line below, replace 'mypicture' with the 
% filename (without extension) and uncomment the line

% \renewcommand{\titleimage}{mypicture}

% (picture proportions: 63 : 20, *.eps format if you use latex+dvips+ps2pdf,
% *.jpg/*.png/*.pdf if you use pdflatex)

% if you want to see BibTeX keys in the references view instead of the symbol,
% uncomment the following line
% \usebibitemtemplate{\insertbiblabel}

% the presentation starts here

\usepackage{floatflt}
\usepackage{listings}

\setcounter{tocdepth}{1}

\title[Probe Specification]{Probe Specification}
\subtitle{SDQ Lerngruppe 2010}
\author{Philipp Merkle, Anne Martens}
\date[04.02.10]{04. Februar 2010}

\institute{Chair for Software Design and Quality}

\begin{document}

\AtBeginSection[]{%
\begin{frame}
\frametitle{Outline}
\tableofcontents[currentsection]
\end{frame}
}% AtBeginSection

\selectlanguage{english}
%title page
\begin{frame}
\titlepage
\end{frame}

%table of contents
\frame{
\frametitle{Outline}
\tableofcontents
}

\section{Motivation}
\subsection{Motivation -- General}
\begin{frame}
\frametitle{Motivation -- General}
\begin{itemize}
%\item Conduct simulations, benchmarks, ...
\item Specify measuring points / track by a model
\item Simplify measurings in simulations, benchmarks, \ldots
\item Integrates persistence framework(s) -- soon
%\item (Messstrecken)
%\item Model-based (driven?) specification of measuring points / tracks
%\begin{itemize}
%\item Multidimensional measurements
%\end{itemize}
\end{itemize}
\end{frame}

\subsection{Motivation -- SimuCom}
\begin{frame}
	\frametitle{Motivation -- SimuCom}
	\begin{itemize}
	\item Current situation
		\begin{itemize}
		\item Where and what to measure hard coded
		\item Persists measurements using SensorFramework
		\item Some things cannot be measured
			\begin{itemize}
			\item e.g. duration between two calls
			\end{itemize}
		\item What if some measurements are not needed?
		\end{itemize}
	\pause
	\item SimuCom with ProbeSpecification
		\begin{itemize}
		\item Where and what defined by a model
		\item Persistence: EDP2 applicable
		\item Allows deactivation of certain measurements
		\end{itemize}
	\end{itemize}
\end{frame}

\section{Overview}
\subsection{Overview}
\begin{frame}
\frametitle{Overview}
\begin{itemize}
\item Probe Specification
	\begin{itemize}
	\item Meta-Model
		\begin{itemize}
		\item Model specifies the measuring track
		\end{itemize}
	\item Framework
		\begin{itemize}
		\item Supports concrete measurings
		\end{itemize}
	\end{itemize}
\pause
\item Pipes-and-Filters
	\begin{itemize}
	\item Filtering, aggregation, persistency of measurements
	\item Meta-Model
		\begin{itemize}
		\item Model specifies a filter chain
		\end{itemize}
	\item Framework
		\begin{itemize}
		\item Provides mechanism to build and use filter chains
		\end{itemize}
	\end{itemize}
\end{itemize}
\end{frame}


\subsection{Probe Specification}
\begin{frame}
\frametitle{Probe Specification}
\begin{itemize}
\item Meta-Model (EMF ECore)
	\begin{itemize}
	\item Models define where and what to measure %(e.g. within a PCM model)
	\item Refers to another meta-model (e.g. PCM) containing ``measurable'' elements
		\begin{itemize}
		\item Measurable (PCM): UsageScenario, ExternalCallAction, EntryLevelSystemCall, ...
		\item Condition: ECore based too %(i.e. measureable model elements inherit EObject)
		\end{itemize}
	%\item Annotates �measureable� model elements (TODO)
	%	\begin{itemize}
	%	\item e.g. annotate ExternalSystemCall to measure response times
	%	\end{itemize}
	%\item Annotated elements: Must inherit EObject
	\end{itemize}
\pause
\item Framework
	\begin{itemize}
	\item Simplifies collecting and calculating measurements %[Vereinfacht Erhebung und Verarbeitung von Messergebnissen]
	%\item Provides abstract measurement methods (``Probes'') %Needs to be adapted to measuring environment (e.g. SimuCom)
	\item Should support various measuring environments
		\begin{itemize}
		\item SimuCom, ProtoCom, \ldots
		%\item Implement custom \emph{ProbeStrategyFactory}
		\end{itemize}
	%\item Extendable / Adjustable
	%	\begin{itemize}
	%	\item Measuring Environment (mentioned above)
	%	\item Persistence Framework (e.g. EDP2)
	%	\end{itemize}
	\item Depends on Pipes-and-Filters
		\begin{itemize}
		\item Delegates results to persistence framework
		\end{itemize}
	\end{itemize}
\end{itemize}
\end{frame}

\subsection{Pipes-and-Filters}
\begin{frame}
\frametitle{Pipes-and-Filters}
\begin{itemize}
\item Meta-Model (EMF ECore)
	\begin{itemize}
	\item Models define concrete filter chains
	\end{itemize}
\pause
\item Framework
	\begin{itemize} 
	\item Provides \emph{PipesAndFiltersManager}
		\begin{itemize}
		\item Simplifies building filter chains
		\item Manages data flow source $\longrightarrow$ sink
		\end{itemize}
	\item Works with arbitrary persistence framework
		\begin{itemize}
		\item SensorFramework, EDP2, \ldots
		\item Implement custom \emph{WriteStrategy}
		\end{itemize}
	\item Independent of Probe Specification $\Rightarrow$ reusable!
	\end{itemize}
%\item Implements Pipes-and-Filters architecture style
%\item Allows filtering
%\item ProbeSpecification uses Pipes-and-Filters
%\item Pipes-and-Filters independent of Probe Specification %Thus reusable
\end{itemize}
\end{frame}

\section{Concepts}
\subsection{Illustration}
\begin{frame}
\frametitle{Concepts -- Illustration}
\begin{itemize}
\item Szenario: Agriculture
	\begin{itemize}
	\item Several deployed sensor nodes which measure \emph{temperature} and \emph{humidity}
	\item Base Station collects node data
	\end{itemize}
\end{itemize}
\begin{figure}
	\centering
		\includegraphics[scale=0.28]{gfx/concepts_illustration.png}
	\label{fig:mm_overview}
\end{figure}
\end{frame}

\subsection{Probe}
\begin{frame}
\frametitle{Probe -- ``Messf\"uhler''}
\begin{columns}[t]
\column{.8\textwidth}
%\begin{floatingfigure}[r]{0.3\textwidth}
%   \includegraphics[scale=0.3]{gfx/illustration_probe.png}
%\end{floatingfigure}
\begin{itemize}
\item Measures a single value + unit pair (JScience: ``Measure'')
	\begin{itemize}
	\item A measured value/unit pair is called ProbeSample
	\end{itemize}
\item Specifies \textbf{what} to measure, depending on probe type
	\begin{itemize}
	\item Knows how to take a measurement
	\end{itemize}
\item Probe types
	\begin{itemize}
	\item Current Time, CPU State, CPU Demand, \ldots
	\end{itemize}
\end{itemize}
\column{.2\textwidth}
\begin{figure}
	\centering
		\includegraphics[scale=0.3]{gfx/illustration_probe.png}
\end{figure}
\end{columns}
\end{frame}

\subsection{ProbeSet}
\begin{frame}
\frametitle{ProbeSet~--~``Messpunkt''}
\begin{columns}[t]
\column{.8\textwidth}
%\begin{floatingfigure}[r]{0.3\textwidth}
%   \includegraphics[scale=0.2]{gfx/illustration_probeset.png}
%\end{floatingfigure}
\begin{itemize}
\item Encapsulates one or several Probes
\item Measures a tuple of value + unit pairs
	\begin{itemize}
	\item Dependent on contained probes
	\item Multidimensional measurements possible
	\item A set of measures is called ProbeSetSample
	\end{itemize}
\item Specifies \textbf{where} to take a measurement 
\item Example ProbeSet: $(Current Time, CPU State)$
%\item	Measuring the state disregarding the time probably useless, Stichwort Auslastung (TODO)
\end{itemize}
\column{.2\textwidth}
\begin{figure}
	\centering
		\includegraphics[scale=0.2]{gfx/illustration_probeset.png}
\end{figure}
\end{columns}
\end{frame}

\subsection{Calculator}
\begin{frame}
\frametitle{Calculator}
\begin{itemize}
\item Motivation: Measure (time) differences % "Main" motivation!?
\item Input: ProbeSetSample(s)
	\begin{itemize}
	\item i.e. the measurements from one or several ProbeSets
	\end{itemize}
\item Output: The measuring result (value + unit)
\item Example: CPUStateCalculator
\item Trivial calculator: Pass through
\item Calculator types
	\begin{itemize}
	\item Response Time, Waiting Time, CPU Demand, \ldots
	\end{itemize}
\end{itemize}
\end{frame}

\begin{frame}
\frametitle{Calculator: Discussion}
\begin{itemize}
\item Aren't the ProbeSetSamples already results? % TODO: Is the ... not already...
	\begin{itemize}
	\item Consider a response time measurement
		\begin{itemize}
		\item 2 ProbeSets needed: At start position, at end position (e.g. of a service call)
		\item Resulting ProbeSetSamples: $(time_{start})$, $(time_{end})$
		\item Calculating $time_{end}-time_{start}$ produces result
		\end{itemize}
	\end{itemize}
\end{itemize}
\end{frame}

\section{Meta-Model}
\subsection{Overview}
\begin{frame}
\frametitle{Meta-Model -- Overview}
\begin{figure}
	\centering
		\includegraphics[scale=0.3]{gfx/mm_overview2.png}
	\label{fig:mm_overview}
\end{figure}
\begin{itemize}
\pause
\item Why a \emph{Position} attribute?
	\begin{itemize}
	\item Consider e.g. an \emph{ExternalCallAction}: Position specifies whether to measure before or after the call
	\end{itemize}
\end{itemize}
\end{frame}

\subsection{Probes}
\begin{frame}
\frametitle{Meta-Model -- Probes}
\begin{figure}
	\centering
		\includegraphics[scale=0.3]{gfx/mm_probes.png}
	\label{fig:mm_overview}
\end{figure}
\end{frame}

\subsection{Calculators}
\begin{frame}
\frametitle{Meta-Model -- Calculators}
\begin{figure}
	\centering
		\includegraphics[scale=0.3]{gfx/mm_calculators.png}
	\label{fig:mm_overview}
\end{figure}
\end{frame}

\subsection{Pipes-and-Filters}
\begin{frame}
\frametitle{Meta-Model -- Pipes-and-Filters}
\begin{figure}
	\centering
		\includegraphics[scale=0.3]{gfx/mm_pipesandfilters.png}
	\label{fig:mm_overview}
\end{figure}
\end{frame}

\subsection{Usage Example: SimuCom}
\begin{frame}
\frametitle{Usage Example: SimuCom}
\begin{figure}
	\centering
		\includegraphics[scale=0.2]{gfx/transformation.png}
	\label{fig:mm_overview}
\end{figure}
\end{frame}

\section{Framework}
\subsection{ProbeStrategies}
\begin{frame}
\frametitle{ProbeStrategies}
\begin{columns}[t]
\column{.3\textwidth}
\begin{figure}
	\centering
		\includegraphics[scale=0.3]{gfx/probeStrategies.png}
\end{figure}
\column{.6\textwidth}
\begin{itemize}
\item ProbeStrategy
	\begin{itemize}
	\item Concrete measuring method
		\begin{itemize}
		\item Wall clock time:\\{\tiny\emph{System.currentTimeMillis()}}
		\item Simulation time:\\{\tiny\emph{SimulationControl.getCurrentSimulationTime()}}
		\end{itemize}
	\vspace{0.16\textheight}
	\item Specific implementation for each measuring environment
	\end{itemize}
\item Different implementations for different environments
\end{itemize}
\end{columns}
\end{frame}

\begin{frame}
\frametitle{ProbeStrategyFactory}
\begin{columns}[t]
\column{.4\textwidth}
\begin{figure}
	\centering
		\includegraphics[scale=0.3]{gfx/probeStrategyFactory2.png}
	\label{fig:SequenceDiagram1}
\end{figure}
\column{.5\textwidth}
\begin{itemize}
\item ProbeStrategyFactory
	\begin{itemize}
	%\item Encapsulates ProbeStrategy creation
	\item Bundles ProbeStrategies for specific measuring environments
	\item One factory for each measuring environment
	%\item Inherits abstract class AProbeStrategyFactory
	%\item Creates concrete ProbeStrategies
	%	\begin{itemize}
	%	\item Thus: Provides a set of ProbeStrategies
	%	\end{itemize}
	%\item One factory for each measuring environment
	\end{itemize}
\end{itemize}
\end{columns}
\end{frame}

\subsection{Blackboard}
\begin{frame}
\frametitle{Blackboard}
\begin{itemize}
\item Receives measurements
\item Notifies calculators of new measurements
	\begin{itemize}
	\item Observer Pattern
	\end{itemize}
\item \emph{Context aware}
	\begin{itemize}
	\item Simulation: Which simulated user raised the measurement?
	\item Code instrumentation: Which thread caused the measurement?
	\end{itemize}
\end{itemize}
\end{frame}

\subsection{Blackboard -- Context awareness}
\begin{frame}
\frametitle{Blackboard -- Context awareness}
\begin{itemize}
\item Implemented by framework class \emph{RequestContextID}
	\begin{itemize}
	\item Represents context
	\end{itemize}
\item Present
	\begin{itemize}
	\item Context represented simply by a String 
		\begin{itemize}
		\item e.g. \emph{UserID} or \emph{ThreadID}
		\end{itemize}
	\item Encapsulated by class \emph{RequestContextID}
	\item Does not support forks
	\end{itemize}
\pause
\item Planned
	\begin{itemize}
	\item Context represented by a \emph{stack} of Strings
	\item Enables measuring e.g. response time within forks
	\end{itemize}
\end{itemize}
\end{frame}

\subsection{Blackboard -- Interaction}
\begin{frame}
\frametitle{Blackboard -- Interaction (Case 1)}
\begin{itemize}
\item Case 1: Unary Calculator
\end{itemize}
\begin{figure}
	\centering
		\includegraphics[scale=0.3]{gfx/SequenceDiagram1.png}
	\label{fig:SequenceDiagram1}
\end{figure}
\pause
\begin{itemize}
\item Not depicted: \lstinline[]{addProbeSample()} and notify() has parameter \emph{ProbeSetSample}
\end{itemize}
\end{frame}

\begin{frame}
\frametitle{Blackboard -- Interaction (Case 2)}
\begin{itemize}
\item Case 2: Binary Calculator, MeasuringEnvironment adds start ProbeSetSample to Blackboard
\end{itemize}
\begin{figure}
	\centering
		\includegraphics[scale=0.3]{gfx/SequenceDiagram3.png}
	\label{fig:SequenceDiagram1}
\end{figure}
\begin{itemize}
\pause
\item ResponseTimeCalculator must wait for end ProbeSetSample (Case 3)
\end{itemize}
\end{frame}

\begin{frame}
\frametitle{Blackboard -- Interaction (Case 3)}
\begin{itemize}
\item Case 3: Binary Calculator, MeasuringEnvironment adds end ProbeSetSample to Blackboard
\end{itemize}
\begin{figure}
	\centering
		\includegraphics[scale=0.3]{gfx/SequenceDiagram2.png}
	\label{fig:SequenceDiagram1}
\end{figure}
\end{frame}

\subsection{Blackboard -- Garbage Collection}
\begin{frame}
\frametitle{Blackboard -- Garbage Collection}
\begin{itemize}
\item Obsolete ProbeSetSamples occupies memory
	%\begin{itemize}
	%\item Obsolete $\Leftrightarrow$ corresponding Calculator already generated an output, based on the ProbeSetSample
	%\end{itemize}
\item Garbage Collection needed
\pause
\item Currently: Basic implementation
	\begin{itemize}
	\item Initial lifetime = 1 for each ProbeSetSample
	\item Invoking \emph{getProbeSetSample()} decreases lifetime by 1
	\item Lifetime == 0 $\Rightarrow$ Garbage
	\end{itemize}
\pause
\item Desireable -- future work
	\begin{itemize}
	\item Get rid of side effects within \emph{getProbeSetSample()}
	\end{itemize}
\end{itemize}
\end{frame}

\section{Demonstration}
\begin{frame}
\frametitle{Demonstration}
\begin{itemize}
\item Live Demo: Test Case ``CalculatorAndPipesTest''
\end{itemize}
\begin{figure}
	\centering
		\includegraphics[scale=0.13]{gfx/CalculatorAndPipeTest.png}
	\label{fig:SequenceDiagram1}
\end{figure}
\end{frame}

\section{References}
\begin{frame}
\frametitle{References, Further Information}
\begin{itemize}
\item SPEEcl Praktikum 2009
	\begin{itemize}
	\item Anforderungsanalyse -- version 0.2 is final
	\item Review Anleitung
	\item Abschlusspr\"asentation
	\end{itemize}
\item Javadoc -- there are extensive package documentations
\item All documents resides in the SVN:\\{\tiny /code/Palladio.ProbeSpecification/trunk/de.uka.ipd.sdq.probespec.framework/doc}
\end{itemize}
\end{frame}

\begin{frame}
\frametitle{Omitted}
\begin{itemize}
\item Pipes-and-Filter
	\begin{itemize}
	\item PipesAndFiltersManager
	\item Recorder: RawRecorder vs. AggregationRecorder
	\item MetaDataInit: Provides WriteStrategies details on the measurement
	\end{itemize}
\end{itemize}
\end{frame}

\end{document}
